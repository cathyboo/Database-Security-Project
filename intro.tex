\section{Introduction}

Databases and database systems have become a major part of everyday life in modern society.  Nearly everything we interact with and depend on have some kind of dependence on a database \ref{ElmNav}.  From taking money out of the bank to paying tax to ordering products on-line.  For decades at this point nearly every business has made use of some kind of database technology to manage employee records, customer details, product and supplier details.  These are the critical functions of a business and yet they are the simplest and oldest uses of a database.  With the advent of web based applications and information systems technology, the potential use of databases to businesses and, consequently, the reliance of businesses on them has exploded.  Never before has security been such a major issue.  Cloud storage and applications open up efficient space and processing power to smaller businesses and socail organisations who would not have had the turnover for the cost of a traditional DBMS housed on-site.  However with this technology a whole new set of security risks have been introduced by exposing companies private data to the potentially public world of the world wide web.  

Security threats can be roughly categorised into the following three areas; 1) unauthorized data observation, 2) incorrect data modification, and 3) data unavailability.  All businesses and organisations may suffer heavy financial and reputational losses due to unauthorized data observation. Incorrect modifications of data, either intentional or unintentional, result in an incorrect database state and the use of incorrect data may in turn lead to heavy losses.  When data is unavailable, crucial information for the proper functioning of a business or organisation is not readily available when needed \ref{BerSand}, just ask anyone who had an Ulster bank account last year how annoying that is!  

Therefore in order for a database to be considered secure the following requirements must be met.  1) Data privacy protection data against unauthorized disclosure, sometimes referred to as secrecy or confidentiality.  2) Data integrity refers to the prevention of unauthorized and improper data modification.  3) Data availability refers to the prevention and recovery from hardware and software errors or physical damage and from malicious data access denials making the database system unavailable. These three requirements arise in practically all application environments \ref{BerSand}&\ref{Thur} and will be dealt with in more detail later in this paper.


\begin{thebibliography}{}

% book
\bibitem{ElmNav}  Ramez Elmasri and Shamkant B. Navathe, {\it Fundementals of Database Systems, 4th Edition}, Pearson (2004)

% paper
\bibitem{BerSand}  Elisa Bertino and Ravi Sandhu, {\it Database Security - Concepts, Approaches and Challenges} IEEE Transactions on Dependable and Secure Computing {\bf 2},  (2005)

% book
\bibitem{Thur}  Bhavani Thuraisingham, {\it Database and Applications Security}, Auerbach (2005)

% paper
\bibitem{Shul}  Amichai Shulman, {\it Top Ten Database Security Threats: How to Mitigate the Most Significant Database Vulnerabilities}, Imperva Website

\bibitem{Ora}  Sumit Jeloka, Don Gosselin and Richard Smith, {\it Oracle\textregistered Database Security Guide, Release 2}, Oracle and/or its affiliates (2012)

\end{thebibliography}
